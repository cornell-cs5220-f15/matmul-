\documentclass[11pt]{article}
\usepackage{geometry}
\geometry{letterpaper}
\geometry{margin=0.75in}


\usepackage[utf8]{inputenc}
\usepackage{listings}
\usepackage{xcolor}
% credit: http://tex.stackexchange.com/questions/68091/how-do-i-add-syntax-coloring-to-my-c-source-code-in-beamer
% https://en.wikibooks.org/wiki/LaTeX/Source_Code_Listings
\lstset{language=C,
        basicstyle=\ttfamily,
        keywordstyle=\color{blue}\ttfamily,
        stringstyle=\color{red}\ttfamily,
        commentstyle=\color{green}\ttfamily,
        morecomment=[l][\color{magenta}]{\#}
}

\title{\textbf{CS 5220 Project 1 Initial Report}}
\author{Stephen McDowell (sjm324)\\ Guantian Zheng (gz94)\\ Jingyang Wang (jw598)}
\date{\textbf{September 2015}}

\usepackage{natbib}
\usepackage{graphicx}
\usepackage{pdfpages}

\newcommand{\tab}{\hspace*{2em}}
\newcommand{\norm}[1]{\lVert#1\rVert}
\newcommand{\sub}{\textsubscript}
\newcommand{\Depth}{2}
\newcommand{\Height}{2}
\newcommand{\Width}{2}

\usepackage{mathtools}
\DeclarePairedDelimiter\ceil{\lceil}{\rceil}
\DeclarePairedDelimiter\floor{\lfloor}{\rfloor}

\usepackage{relsize}

\def\wl{\par \vspace{\baselineskip}}

\makeatletter
\renewcommand{\maketitle}{\bgroup\setlength{\parindent}{0pt}
\begin{flushleft}
  {\Large \textsc{\@title}}\newline
  \textsc{\@author}
  \rule{\textwidth}{1pt}
\end{flushleft}\egroup
}
\makeatother

\usepackage{fancyhdr}
\pagestyle{fancy}
\rhead{\textsc{Group 018} \thepage}
\renewcommand{\headrulewidth}{0pt}
\setlength{\headheight}{0.5in}

\title{CS 5220: Project 1 Initial Report}
\author{Group 018: Guantian Zheng (gz94), Stephen McDowell (sjm324)}

\begin{document}
\thispagestyle{empty}
\maketitle

\section{Introduction}

\section{Permutations of Loop Variables}

%\includepdf[pages=-, width=\textwidth]{timing-perm}

% \begin{figure}[h!]
% \centering
% \includegraphics[width=\textwidth]{timing-perm.jpg}
% \caption{Performance of \texttt{basic} under different permutations of i, j, k.}
% \label{fig:timing-perm}
% \end{figure}

We experimented with different permutations of the three loop variables \texttt{i}, \texttt{j} and \texttt{k}. Apparently, \texttt{j, k, i} (starting from the outermost loop) gains the most advantage by reading in continuous blocks of matrix \texttt{C} and \texttt{A}. The loops are as follows:

\begin{lstlisting}
    for(j = 0; j < M; ++j) {
        for(k = 0; k < M; ++k) {
            double b_kj = B(k, j);
            for(i = 0; i < M; ++i) {
                C(i, j) += A(i, k) * b_kj;
            }
        }
    }
\end{lstlisting}

\section{Tuning the Block Size}

Building on previous results, we decided to adopt the \texttt{j-k-i} loop for per-block computation (\texttt{basic\_dgemm}), while searching for an appropriate block size. 

% \begin{figure}[h!]
% \centering
% \includegraphics[width=\textwidth]{timing-blocked-1.jpg}
% \caption{Performance of original and reordered \texttt{blocked} with block sizes under 256.}
% \label{fig:timing-blocked-1}
% \end{figure}

Looks like reordered \texttt{blocked} with block size 256 takes the lead. Let's try with larger sizes:

% \begin{figure}[h!]
% \centering
% \includegraphics[width=\textwidth]{timing-blocked-2.jpg}
% \caption{Performance of original and reordered \texttt{blocked} with block sizes under 2048.}
% \label{fig:timing-blocked-2}
% \end{figure}

The Gflops of \texttt{blocked\_perm} keeps rising until size 2048, which means \texttt{j-k-i} looping blocked implementation with size 1024 is by far the optimal solution for our test cases.









\iffalse

\section{Conclusion}
``I always thought something was fundamentally wrong with the universe'' \citep{adams1995hitchhiker}

\bibliographystyle{plain}
\bibliography{references}

\fi

\end{document}
