\subsection{Three Tiered Blocks}
We attempted to extend our previous blocking implementations with multiple
levels of blocking. The original idea behind this method was to have 3 layers 
of blocking that each corresponded to a level of caching. Based on the results 
of the membench program, we estimated the L3 cache to be 15MB, the L2 cache to 
be 256kB, and the L1 cache to be 64kB. Each of these caches could thus store 
square matrices with a dimension of 1402, 181, and 45, respectively. Since 
these weren't very nice numbers we decided to round down to neat powers of two 
and set the block sizes at 1024, 128, and 32, respectively. While this 
certainly yielded a performance boost over the basic version, this 
implementation still fell well short of other kernels.

A second attempt to optimize this method was by adding padding to the lowest 
level of blocking to help create more uniform memory access patterns. In doing 
this we also experimented with making the lowest level of blocking smaller to 
prevent too many unnecessary zeros being calculated. This yielded some 
performance boost, but nothing significant. We also played around with the 
size of the two other blocking levels as well as only doing two levels of 
blocking, but most of these attempts only hindered performance.

Our final attempt to increase performance was to incorpate the results from 
previous blocking attempts. These other attempts always found that big block 
sizes, larger than 128, yielded the best results. For this reason we increased 
the all the block sizes and only went down one level of blocking if the matrix 
at the current blocking level was not a square. This yielded the best and most 
consistent results of all the attempts.