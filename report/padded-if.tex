\subsection{Dynamic Padded Blocks}
\figref{padded-sweep} shows that the performance of the \ttt{padded\_blocked}
kernel performs very well when the matrix size is a multiple of the block size.
On the other hand, the kernel performs poorly when matrix size is just over a
multiple of the block size. This is consistent with our intuition. When the
matrix size is just over a multiple of the block size, the final set of blocks
in the multiplication are mostly empty, so the algorithm performs superfluous
computation.

The \ttt{padded\_if} kernel is a hybrid between the \ttt{padded\_blocked} and
\ttt{big\_blocked} kernel. When multiplying two blocks of size
$\ttt{BLOCK\_SIZE} \times \ttt{BLOCK\_SIZE}$, the kernel uses the optimized
multiplication from the \ttt{padded\_blocked} kernel. However, when the size of
the blocks being multiplied are smaller than $\ttt{BLOCK\_SIZE} \times
\ttt{BLOCK\_SIZE}$, the kernel uses the unoptimized multiplication from the
\ttt{big\_blocked} kernel. Thus, the \ttt{padded\_if} kernel does not perform
superfluous computation, but it does have some overhead in dispatching between
two different loops.
