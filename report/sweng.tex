\section{Software Engineering}\label{sec:sweng}
The development of an efficient matrix multiplication kernel is a complicated
task that involves a lot of rapid prototyping, and the efficiency with which we
can develop kernels is limited by how easy it is to write correct code and by
the overhead of creating, modifying, and running kernels. In this section, we
describe a couple of the software engineering principles we adhered to that
increased the overall productivity of kernel development. Concretely, we
discuss our use of helper functions and unit tests as well as our improvements
to the build system.

While not directly related to kernel optimization, we feel that our
organization and principled software development is a significant achievement
that indirectly has a large impact on the quality of kernels written.

\subsection{Helper Functions and Unit Tests}
Matrix multiplications involve a lot of low-level pointer arithmetic, and in a
programming language like C, erroneous pointer arithmetic or invalid memory
access can be \emph{very difficult} to debug. Moreover, matrix multiplication
involves operations on a combination of row-major and column-major matrices
using various complex access patterns involving many nested layers of blocking
transposing and copying which greatly increase the likelihood of incorrect
code. To combat the likelihood of programming error and reduce the time spent
debugging, we developed a modular set of universal helper functions accompanied
with a set of unit tests.

For example, one unavoidable part of matrix multiplication is array indexing.
We store all two-dimensional matrices as one-dimensional C arrays which means
that indexing a matrix involves computing an offset. This offset computation
depends on whether the array is stored in row-major  or column-major order.
Things get even more complicated when indexing into a matrix that is embedded
in a larger matrix or when transposing matrices converting them from
column-major to row-major.

Rather than manually inlining the index calculations, we instead developed two
helper functions that compute array offsets for column-major and row-major
matrices. Thus, if we want to access the $i,j$\textsuperscript{th} element of a
column-major matrix $A$, we don't have to struggle to choose between \ttt{A[j +
N*i]}, \ttt{A[i + N*j]}, \ttt{A[j + M*i]}, or \ttt{A[i + M*j]}. Instead, we
write \ttt{A[cm(N, M, i, j)]}. Similarly, we access row-major arrays using
\ttt{A[rm(N, M, i, j)]}.

We have written similar helper functions for transposing, copying, and clearing
arrays and subarrays. Each helper function is verified with a set of unit
tests, and each function is inlined to avoid any cost of function calls.
Overall, this modularity and verification has greatly reduced a large number of
common errors that take a significant amount of time to debug allowing us to
spend more time focused on development and optimization.

\paragraph{Enhanced Makefiles}
\todo{
  Discuss the use of helper functions and test cases, the modification of the
  Makefile, and the use of a single runner pbs file. Argue that this was
  significant work and had positive impact.
}
