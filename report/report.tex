   
\documentclass[11pt]{article}
\usepackage{amsmath,amsthm,verbatim,amssymb,amsfonts,amscd, graphicx}
\usepackage{graphics}
\topmargin0.0cm
\headheight0.0cm
\headsep0.0cm
\oddsidemargin0.0cm
\textheight23.0cm
\textwidth16.5cm
\footskip1.0cm
\theoremstyle{plain}
\newtheorem{theorem}{Theorem}
\newtheorem{corollary}{Corollary}
\newtheorem{lemma}{Lemma}
\newtheorem{proposition}{Proposition}
\newtheorem*{surfacecor}{Corollary 1}
\newtheorem{conjecture}{Conjecture} 
\newtheorem{question}{Question} 
\theoremstyle{definition}
\newtheorem{definition}{Definition}

\begin{document}
\title{CS 5220\\ Project 1 - Matrix Multiplication}
\author{Weici Hu(wh343)\\ Sheroze Sheriffdeen(mss385)\\ Qinyu Wang(qw78)}
\maketitle

\section{Introduction}
In this project, we tried several methods to fine-tune square matrix multiplication.
Based on the \texttt{dgemm\_blocked.c}, we tried unrolling index, modifying loop sequence to take advantage of SSE, and experimenting with different optimization flags.

\section{Optimization}
\subsection{Block Multiplication with Multiple Block Sizes}
\subsubsection{Approach}
//Describe what we did here.
\subsubsection{Results}
//Add graphs here.

\subsection{Block Multiplication with Manual Loop Unrolling}
\subsubsection{Approach}
//Describe what we did here.
\subsubsection{Results}
//Add graphs here.

\subsection{AVX Instructions}
\subsubsection{Approach}
//Describe what we did here.
\subsubsection{Results}
//Add graphs here.

\subsection{Compiler Optimization Flags}
\subsubsection{Approach}
//Describe what we did here.
\subsubsection{Results}
//Add graphs here.

\section{Next Steps}
\subsection{Copy Optimization}





 
 
\end{document}