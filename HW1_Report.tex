%%%%%%%%%%%%%%%%%%%%%%%%%%%%%%%%%%%%%%%%%
% Large Colored Title Article
% LaTeX Template
% Version 1.1 (25/11/12)
%
% This template has been downloaded from:
% http://www.LaTeXTemplates.com
%
% Original author:
% Frits Wenneker (http://www.howtotex.com)
%
% License:
% CC BY-NC-SA 3.0 (http://creativecommons.org/licenses/by-nc-sa/3.0/)
%
%%%%%%%%%%%%%%%%%%%%%%%%%%%%%%%%%%%%%%%%%

%----------------------------------------------------------------------------------------
%	PACKAGES AND OTHER DOCUMENT CONFIGURATIONS
%----------------------------------------------------------------------------------------

\documentclass{article}	 % A4 paper and 11pt font size
\usepackage[english]{babel} % English language/hyphenation
\usepackage[protrusion=true,expansion=true]{microtype} % Better typography
\usepackage{amsmath,amsfonts,amsthm} % Math packages
\usepackage[svgnames]{xcolor} % Enabling colors by their 'svgnames'
\usepackage{booktabs} % Horizontal rules in tables
\usepackage[margin=1.0in]{geometry}
\usepackage{xcolor}
\usepackage{float}
\usepackage{graphicx}
\usepackage{subcaption}
\usepackage{lastpage} % Used to determine the number of pages in the document (for "Page X of Total")

%----------------------------------------------------------------------------------------
%	TITLE SECTION
%----------------------------------------------------------------------------------------

%----------------------------------------------------------------------------------------

\begin{document}

\begin{center}
    \Large
    \textbf{Assignment 1: \\ Matrix Multiplication}
    
    \vspace{0.4cm}
    \large
        
    \vspace{0.4cm}
    Lara Backer \\ Unmukt Gupta \\ Edward Tremel

\end{center}

%----------------------------------------------------------------------------------------
%  CONTENTS
%----------------------------------------------------------------------------------------

\section{Overview}

Matrix multiplication is one of the most fundamental computations required to solve linear systems, and as such is of vital importance in numerous computational applications. Therefore, reducing computation time for matrix multiplication calculations is an obvious method for increasing the efficiency of many solvers. \\

In this paper, we will be discussing the implementation and results from multiple methods for optimizing matrix multiplication of equation \ref{eq:mm}, with matrix dimensions M, N, and K. Results are compared to often-used matrix multiplication routines, such as those present in BLAS, a basic blocked multiplication method, and a naive matrix multiplication routine. 

\begin{equation}
C (M x N) = A (M x K) * B (K x N)
\label{eq:mm}
\end{equation}

% Figure insert formatting - for use on timing plots
%\begin{figure}[H]
%\centering
%  \centering
%  \includegraphics[width=.6\linewidth]{mesa_timings}
%  \caption{Speed-up}
%  \label{fig:mesatimings}
%  \end{figure}

\section{Optimization Techniques}

\subsection{Blocking}
Blocking techniques take advantage of the temporal locality of inner loops. Each chunk, or 'block', of a matrix, is chosen so that the code loads the necessary block into cache for each loop, discarding the block when the loop is finished. By knowing the cache level sizes, the block sizes can be chosen to fit within the cache and avoid cache misses, which will decrease code performance. 

The head node on the totient cluster used to run this code is a Intel Xeon E5-2620 v3 processor. 

\subsubsection{Level 2 cache}
The processor level 2 cache size is 256KB. 

\subsubsection{Level 1 Cache}
The processor level 1 cache size is 32KB. A square block containing double precision numbers would thus be able to be ___ size and still fit in the level 1 cache. 


\subsection{Copy Optimization}
\subsection{SSE Directives}

Intel intrinsic functions were tested ...

\subsection{Compiler Flags}
Compiler flags that were found to improve performance were: \\
\begin{itemize}
\item -O2: enables optimizations for speed, such as compiler vectorization and loop transformations.
\item -no-prec-div
\item -funroll-loops
\item -ipo

\end{itemize}

\section{Conclusion}

%----------------------------------------------------------------------------------------

\end{document}