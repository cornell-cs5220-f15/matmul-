\documentclass[11pt]{amsart}
\usepackage{geometry}                % See geometry.pdf to learn the layout options. There are lots.
\geometry{letterpaper}                   % ... or a4paper or a5paper or ... 
%\geometry{landscape}                % Activate for for rotated page geometry
\usepackage[parfill]{parskip}    % Activate to begin paragraphs with an empty line rather than an indent
\usepackage{graphicx}
\usepackage{amssymb}
\usepackage{epstopdf}
\DeclareGraphicsRule{.tif}{png}{.png}{`convert #1 `dirname #1`/`basename #1 .tif`.png}

\title{August 25th, 2015 Lecture Questions}
\author{Elliot Cartee}
%\date{}                                           % Activate to display a given date or no date

\begin{document}
\maketitle
\subsection*{Question 1}
To come up with a simple model, we will make the following assumptions:
\begin{itemize}
	\item the class is distributed in $n$ rows and $m$ columns
	\item the time ($t_\text{init}$) to start the process does not depend on the number of people
	\item the time to compute rows is proportional to the number of people in each row (i.e. the number of columns)
	\item the time for the instructor to gather counts is proportional to the number of rows
	\item the time for the instructor to do the arithmetic is proportional to the number of rows
\end{itemize}
Thus we get that the total time to count attendance is:
$$
t_\text{total} = t_\text{init} + a*m + b*n + c*n
$$

\subsection*{Question 2}
In the serial case, we now assume that:
\begin{itemize}
	\item the class is distributed in $n$ rows and $m$ columns
	\item the time to start is now $0$ since the person counting already knows the counting procedure and does not need to explain it to themselves
	\item the time for the instructor to count attendance is proportional to the total number of people
\end{itemize}
Thus we get that the total time for the serial case is:
$$
t_\text{total} = a*m*n
$$
Note that this $a$ is the same as in Question 1

\subsection*{Question 3}

The parallel method takes less time when:
$$
t_\text{init} + a*m + b*n + c*n < a*m*n
$$
Now we will estimate some parameter values:
\begin{itemize}
\item $t_\text{init}$ = time to explain procedure     = 30sec

\item $a$      = time to count each person     = 0.5sec/person

\item $b$      = time to gather count from row = 1 sec/person

\item $c$      = time to do arithmetic         = 1 sec/row
\end{itemize}
Then the parallel method takes less time when:

$$
30 + 0.5*m + 2*n < 0.5*m*n
$$

If we further assume that the number of rows and columns are the same (i.e. m=n), then
the parallel method takes less time when:
$$
30 + 2.5*n < 0.5*n^2
$$
this occurs whenever $n > 10.6394$, in other words whenever there are at least 11 rows and columns, and thus at least 121 people.



\end{document}  